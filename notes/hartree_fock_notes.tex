\documentclass[draft]{article}

% - Style
\usepackage{base}

% - Title
\title{Hartree Fock Notes}
\author{Tom Ross}
\date{Date}

% - Headers
\pagestyle{fancy}
\fancyhf{}
\rhead{\theauthor}
\chead{}
\lhead{\thetitle}
\rfoot{\thepage}
\cfoot{}
\lfoot{}

\begin{document}

\newcommand{\molecule}[1][]{\Psi\indices{#1}}
\newcommand{\orbitalspin}[1][]{\chi\indices{#1}}

\newcommand{\orbital}[1][]{\Omega\indices{#1}}
\newcommand{\orbitalpw}[1][]{\omega\indices{#1}}

\newcommand{\electron}[1][]{\Phi\indices{#1}}
\newcommand{\sturmian}[1][]{\phi\indices{#1}}
\newcommand{\laguerre}[1][]{\xi\indices{#1}}

\newcommand{\harmonic}[1][]{Y\indices{#1}}
\newcommand{\irrep}[1][]{C\indices{#1}}

\newcommand{\spin}{\sigma}
\newcommand{\parity}{\pi}
\newcommand{\proj}{m}
\renewcommand{\ang}{\ell}

\section{Frozen-Core Hartree-Fock Potential}
\label{sec:frozen-core-hartree}

We consider a molecule $\molecule[^{\lr{\spin \pi \proj}}]$ which has spin
$\spin$, parity $\pi$, projection $\proj$ and is composed of $N$ occupied
spin orbitals $\lrset{\orbitalspin[^{\lr{\proj}}_n]}_{n = 1}^{N}$.  To ensure
the total spin of the molecule is a conserved quantity, we assume that the spin
orbitals are formed from a set of doubly-occupied spatial orbitals,
$\lrset{\orbital[^{\lr{\proj}}_n]}_{n = 1}^{N/2}$.  Furthermore, we suppose that
each of these spatial orbitals can be expanded in partial waves
\begin{equation*}
  \orbital[^{\lr{\proj}}_n]\lr{r, \theta, \phi}
  =
  \sum_{\ang = 0}^{\infty}
  \orbitalpw[_{n \ang}]\lr{r}
  \harmonic[_{\ang \proj}]\lr{\theta, \phi}
\end{equation*}
where $\orbitalpw[_{n \ang}]$ are referred to as the partial waves, and
$\harmonic[_{\ang \proj}]$ are the spherical harmonics.

~

We consider the case where an electron $\electron[^{\lr{\spin'}}]$, with spin
$\spin'$, interacts with this molecule.
To do so, we suppose that the electron can be expanded in terms of a basis
the electronic state can be expanded in terms of a basis
$\lrset{\sturmian[_{\alpha}] =
  \laguerre[_{k_{\alpha} \ang_{\alpha}}]\lr{r}
  \harmonic[_{\ang_{\alpha} \proj_{\alpha}}]\lr{\theta, \phi}}$
and consider how the basis elements interact with the molecule.

\subsection{Local Coulomb Potential}
\label{sec:coulomb}

Coulombic interaction between basis elements and the electrons in the
molecule give rise to matrix elements of the form
\begin{equation*}
  J_{\alpha \beta}
  =
  \mel*{
    \sturmian[_{\alpha}]
  }{
    \hat{J}
  }{
    \sturmian[_{\beta}]
  }
  =
  2
  \sum_{n = 1}^{N/2}
  \braket*{
    \sturmian[_{\alpha}]
    \orbital[^{\lr{\proj}}_n]
  }{
    \sturmian[_{\beta}]
    \orbital[^{\lr{\proj}}_n]
  }
\end{equation*}
where
\begin{multline*}
  \braket*{
    \sturmian[_{\alpha}]
    \orbital[^{\lr{\proj}}_n]
  }{
    \sturmian[_{\beta}]
    \orbital[^{\lr{\proj}}_n]
  }
  =
  \\
  \sum_{l_{1} = 0}^{\infty}
  \sum_{l_{2} = 0}^{\infty}
  \sum_{\lambda = 0}^{\infty}
  \sum_{\mu = -\lambda}^{\lambda}
  \mel*{
    \laguerre[_{k_{\alpha} \ang_{\alpha}}]
    \orbitalpw[_{n \ang_{1}}]
  }{
    \tfrac
    {
      r_{<}^{\lambda}
    }
    {
      r_{>}^{\lambda + 1}
    }
  }{
    \laguerre[_{k_{\beta} \ang_{\beta}}]
    \orbitalpw[_{n \ang_{2}}]
  }
  \mel*{
    \harmonic[_{\ang_{\alpha} \rho_{\alpha}}]
  }{
    \irrep[_{\lambda \mu}]
  }{
    \harmonic[_{\ang_{\beta} \rho_{\beta}}]
  }
  \mel*{
    \harmonic[_{\ang_{1} \proj}]
  }{
    {\irrep[_{\lambda \mu}]}^{\dagger}
  }{
    \harmonic[_{\ang_{2} \proj}]
  }
\end{multline*}
and where $\irrep[_{\lambda \mu}]$ is the irreducible spherical harmonic.
However, by introducing a potential expanded through partial waves in the
following form
\begin{equation*}
  V\indices{_{\lambda \mu}}\lr{r}
  =
  2
  \sum_{n = 1}^{N/2}
  \sum_{l_{1} = 0}^{\infty}
  \sum_{l_{2} = 0}^{\infty}
  \mel*{
    \harmonic[_{\ang_{1} \proj}]
  }{
    {\irrep[_{\lambda \mu}]}^{\dagger}
  }{
    \harmonic[_{\ang_{2} \proj}]
  }
  \lr[\Bigg]
  {
    \int\limits_{0}^{\infty}
    \orbitalpw[_{n \ang_{1}}]\lr{r'}
    \frac
    {
      \min\lr{r, r'}^{\lambda}
    }
    {
      \max\lr{r, r'}^{\lambda + 1}
    }
    \orbitalpw[_{n \ang_{2}}]\lr{r'}
    r'^{2}
    \dd{r'}
  }
\end{equation*}
we may write
\begin{equation*}
  J_{\alpha \beta}
  =
  \sum_{\lambda = 0}^{\infty}
  \sum_{\mu = -\lambda}^{\lambda}
  \mel*{
    \laguerre[_{k_{\alpha} \ang_{\alpha}}]
  }{
    V\indices{_{\lambda \mu}}
  }{
    \laguerre[_{k_{\beta} \ang_{\beta}}]
  }
  \mel*{
    \harmonic[_{\ang_{\alpha} \rho_{\alpha}}]
  }{
    \irrep[_{\lambda \mu}]
  }{
    \harmonic[_{\ang_{\beta} \rho_{\beta}}]
  }.
\end{equation*}
Note that as a result of constraints on the Wigner 3-j symbol, the partial waves
are non-vanishing only for $\mu = 0$, so only $V\indices{_{\lambda 0}}$ is of
concern.

\subsection{Non-Local Exchange Potential}
\label{sec:exchange}

Fermionic behaviour of electrons introduces a non-local coulombic interaction,
known as the exchange potential.
The matrix elements for this potential are given by
\begin{equation*}
  K_{\alpha \beta}
  =
  \mel*{
    \sturmian[_{\alpha}]
  }{
    \hat{K}
  }{
    \sturmian[_{\beta}]
  }
  =
  \sum_{n = 1}^{N/2}
  \braket*{
    \sturmian[_{\alpha}]
    \orbital[^{\lr{\proj}}_n]
  }{
    \orbital[^{\lr{\proj}}_n]
    \sturmian[_{\beta}]
  }
\end{equation*}
where
\begin{multline*}
  \braket*{
    \sturmian[_{\alpha}]
    \orbital[^{\lr{\proj}}_n]
  }{
    \orbital[^{\lr{\proj}}_n]
    \sturmian[_{\beta}]
  }
  =
  \\
  \sum_{l_{1} = 0}^{\infty}
  \sum_{l_{2} = 0}^{\infty}
  \sum_{\lambda = 0}^{\infty}
  \sum_{\mu = -\lambda}^{\lambda}
  \mel*{
    \laguerre[_{k_{\alpha} \ang_{\alpha}}]
    \orbitalpw[_{n \ang_{1}}]
  }{
    \tfrac
    {
      r_{<}^{\lambda}
    }
    {
      r_{>}^{\lambda + 1}
    }
  }{
    \orbitalpw[_{n \ang_{2}}]
    \laguerre[_{k_{\beta} \ang_{\beta}}]
  }
  \mel*{
    \harmonic[_{\ang_{\alpha} \rho_{\alpha}}]
  }{
    \irrep[_{\lambda \mu}]
  }{
    \harmonic[_{\ang_{2} \proj}]
  }
  \mel*{
    \harmonic[_{\ang_{1} \proj}]
  }{
    {\irrep[_{\lambda \mu}]}^{\dagger}
  }{
    \harmonic[_{\ang_{\beta} \rho_{\beta}}]
  }.
\end{multline*}
A radial potential cannot be introduced to simplify the calculation of these
elements due to the non-locality of the interaction.

\end{document}